% SIDIS Diagram with Gluon Emission (Illustrative for k_T)
\documentclass[tikz]{standalone}
\usepackage{tikz-feynman}
\begin{document}
\begin{tikzpicture}
  \begin{feynman}
    \diagram[
      horizontal=a to b,
      layered layout,
    ] {
      % Incoming lepton
      i1 [particle=\(l\)] -- [fermion] (v1);
      % Outgoing lepton
      (v1) -- [fermion] f1 [particle=\(l'\)];
      % Virtual photon
      (v1) -- [boson, edge label=\(\gamma^*\)] (v2);

      % Incoming nucleon (represented by its constituent quark)
      i2 [particle=\(N\)] -- [opacity=0] (q_entry_nucleon);
      % Gluon emission from quark inside nucleon
      (q_entry_nucleon) -- [fermion] (g_emission_vertex);
      (g_emission_vertex) -- [gluon, edge label=\(g\)] (g_out);
      (g_emission_vertex) -- [fermion, edge label'=\(q(x, k_T)\)] (v2); % Quark after gluon emission, now has k_T
      
      % Proton remnants (simplified)
      (q_entry_nucleon) -- [draw=none, dotted, thick, segment length=2pt, segment sep=2pt, line to] (remnants) -- [opacity=0] f3 [particle=\(X_{remnants}\)];

      % Outgoing quark (fragmenting)
      (v2) -- [fermion] (q_out);
      % Produced hadron
      (q_out) -- [fermion] f2 [particle=\(h\)];
      % Other fragmentation products (simplified)
      (q_out) -- [draw=none, dotted, thick, segment length=2pt, segment sep=2pt, line to] (frag_remnants) -- [opacity=0] f4 [particle=\(X'\)];

      % Blob representing the nucleon
      \node[draw, circle, fit=(i2) (q_entry_nucleon) (g_emission_vertex) (remnants), minimum size=1.7cm, label=below:Nucleon] (nucleon_blob) at ($(i2)!0.5!(g_emission_vertex) + (0,-0.7)$) {};
      % Blob representing fragmentation
      \node[draw, ellipse, fit=(q_out) (f2) (frag_remnants), minimum width=1.5cm, minimum height=1cm, label=above:Fragmentation] (frag_blob) at ($(q_out)!0.5!(f2) + (0,0.25)$) {};
    };
  \end{feynman}
\end{tikzpicture}
\end{document}

