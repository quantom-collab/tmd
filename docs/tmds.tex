\documentclass{article}
\usepackage{amsmath}
\usepackage{amsfonts}
\usepackage{amssymb}

\usepackage[scale=0.8]{geometry}

\title{Understanding Transverse Momentum Distributions and Their Evolution Equations}
\author{Chiara Bissolotti\\
        Physics Division, Argonne National Laboratory \\
        cbissolotti@anl.gov}
\date{\today}

\begin{document}

\maketitle
\tableofcontents
\newpage

\section{Introduction}

Transverse Momentum Distributions (TMDs) are fundamental theoretical tools that provide a three-dimensional picture of the internal structure of
hadrons, such as protons and neutrons. They describe the probability of finding a parton (a quark or a gluon) within a hadron not only with a certain
fraction of the hadron's longitudinal momentum ($x$) but also with a specific momentum component transverse to the hadron's overall direction of
motion ($k_T$ or $k_\perp$). This level of detail goes beyond the `simpler', one-dimensional picture offered by collinear Parton Distribution
Functions (PDFs).

This document aims to provide a detailed understanding of Transverse Momentum Distributions (TMDs), their theoretical underpinnings in Quantum
Chromodynamics (QCD), their relevance in processes like Semi-Inclusive Deep Inelastic Scattering (SIDIS), and the associated formalism.

\subsection{Definition}

An unrenormalized unpolarized TMD, $\hat{f}_i^{h}(x, \mathbf{k}_T)$, is defined as:

\begin{equation}
  \hat{f}_i^{h}(x, \mathbf{k}_T) = \int \frac{d^2 \mathbf{b}_T}{{(2\pi)}^2} e^{-i \mathbf{b}_T \cdot \mathbf{k}_T} \int \frac{db^-}{4\pi}
  e^{-ix P^+ b^-} \langle P | \bar{\psi}(b) \gamma^+ {\mathcal{W}{(b,0)}} \psi(0) | P \rangle
  \label{eq:correlator_definition}
\end{equation}

where $\mathbf{k}_T$ is the intrinsic transverse momentum of parton $i$ in hadron $h$. The hat over the $f$ indicates this is an unrenormalized
quantity, meaning it contains ultraviolet (UV) and rapidity divergences that need to be regulated and renormalized.

\begin{equation}
  b = (0^{+}, b^-, \mathbf{b}_T)
\end{equation}
%
where the displacement of the quark field $\psi$ is not only in the ``$-$'' direction but also in the transverse direction, $\mathbf{b}_T$.

$\mathcal{W}$ is the Wilson line to ensure SU (3) color gauge invariance, and it is understood to be the staple-shape link.

Look also at Ref.\cite{Collins:2011zzd} for a more detailed discussion of TMDs.

\subsection{Collins-Soper Equation}

The Collins-Soper equation describes the evolution of TMDs with respect to the rapidity scale $\zeta$. Rapidity is a convenient variable in
high-energy physics that relates to the longitudinal momentum of a particle. The CS equation arises from the requirement that physical observables
should be independent of the choice of factorization scheme, which separates the hard scattering part of a process from the non-perturbative TMDs.

To derive the CS equation, one typically starts by considering a physical process where TMDs are relevant, such as Drell-Yan lepton pair production
or semi-inclusive deep inelastic scattering (SIDIS). In these processes, the cross-section can be factorized into a hard part, calculable in
perturbative QCD, and TMDs, which describe the non-perturbative structure of the hadrons.

The key insight is that the TMDs themselves depend on two scales: the renormalization scale $\mu$ and a rapidity scale $\zeta$. The CS equation
describes how the TMDs change as $\zeta$ varies. It can be written as:

\begin{equation}
  \frac{d \ln f(x, k_T, \mu, \zeta)}{d \ln \zeta} = K(k_T, \mu)
  \label{eq:CS}
\end{equation}

Here, $f(x, k_T, \mu, \zeta)$ is a generic TMD, and $K(k_T, \mu)$ is the Collins-Soper kernel (also known as the rapidity anomalous dimension). The
kernel $K$ depends on the transverse momentum $k_T$ and the renormalization scale $\mu$. It can be calculated perturbatively in QCD.\@

The derivation of the CS equation involves analyzing the operator definition of TMDs and their behavior under changes in the rapidity scale. This
often involves techniques from effective field theories like Soft-Collinear Effective Theory (SCET), which provides a systematic framework for
separating different momentum scales in QCD processes.

\subsection{Renormalization Group Equations}

In addition to the rapidity evolution, TMDs also evolve with the renormalization scale $\mu$. This evolution is described by standard Renormalization
Group Equations (RGEs), similar to those for collinear Parton Distribution Functions (PDFs). The RGEs for TMDs account for the scale dependence
arising from ultraviolet divergences in the quantum field theory calculations.

The RGE for a TMD $f$ can be written as:

\begin{equation}
  \frac{d f(x, k_T; \mu, \zeta)}{d \ln \mu^2} = P(x, \alpha_s(\mu)) \otimes f(x, k_T; \mu, \zeta)
  \label{eq:RGE}
\end{equation}

Here, $P(x, \alpha_s(\mu))$ represents the splitting functions (analogous to the DGLAP splitting functions for collinear PDFs), and $\otimes$ denotes
a convolution in the appropriate variables (typically $x$ and $k_T$). These splitting functions describe how partons can split into other partons,
changing the momentum distribution. They are calculable in perturbative QCD.\@

The combined effect of the CS equation and the RGEs determines the full evolution of TMDs with energy. These equations are crucial for making precise
predictions for a wide range of high-energy scattering processes and for understanding the three-dimensional structure of hadrons.

\bibliography{biblio}
\bibliographystyle{JHEP}
\end{document}

