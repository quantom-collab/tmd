% SIDIS Handbag Diagram (tikz-feynman)
\documentclass[tikz]{standalone}
\usepackage{tikz-feynman}
\begin{document}
\begin{tikzpicture}
  \begin{feynman}
    \diagram[
    horizontal=a to b,
    layered layout,
    % Adjust spacing if needed
    % horizontal distance=1.5cm,
    % vertical distance=2cm
    ] {
    % Incoming lepton
    i1 [particle=\(l\)] -- [fermion] (v1);
    % Outgoing lepton
    (v1) -- [fermion] f1 [particle=\(l'\)];
    % Virtual photon
    (v1) -- [boson, edge label=\(\gamma^* (Q^2)\)] (v2);

    % Incoming nucleon (represented by its constituent quark)
    i2 [particle=\(N(P)\)] -- [opacity=0] (q_in_N_helper_start); (q_in_N_helper_start) -- [opacity=0] (q_in_N_helper_end) -- [opacity=0]
    (v2_quark_entry_point); (v2_quark_entry_point) -- [fermion, edge label=\(q(x, k_T)\)] (v2);

    % Proton remnants
    (v2_quark_entry_point) -- [opacity=0] (remnants_helper_start);
    (remnants_helper_start) -- [draw=none, dotted, thick, segment length=2pt, segment sep=2pt, line to] (remnants_helper_end) -- [opacity=0] f3 [particle=\(X_{remnants}\)];

    % Outgoing quark (fragmenting)
    (v2) -- [fermion] (q_out);
    % Produced hadron
    (q_out) -- [fermion] f2 [particle=\(h(P_h)\)];
    % Other fragmentation products
    (q_out) -- [draw=none, dotted, thick, segment length=2pt, segment sep=2pt, line to] (frag_remnants_helper) -- [opacity=0] f4 [particle=\(X'\)];

    % Blob representing the nucleon
    \node[draw, circle, fit=(i2) (v2_quark_entry_point) (remnants_helper_start) (remnants_helper_end), minimum size=1.5cm, label=below:Nucleon] (nucleon_blob) at ($(i2)!0.5!(v2_quark_entry_point) + (0,-0.5)$) {};
    % Blob representing fragmentation
    \node[draw, ellipse, fit=(q_out) (f2) (frag_remnants_helper), minimum width=1.5cm, minimum height=1cm, label=above:Fragmentation] (frag_blob) at ($(q_out)!0.5!(f2) + (0,0.25)$) {};
    };
    % Add labels for vertices if needed, e.g. for interaction points
    % \node [above=0.1cm of v1] {\(e\gamma e\)};
    % \node [above=0.1cm of v2] {\(q\gamma q\)};
  \end{feynman}
\end{tikzpicture}
\end{document}

